\documentclass[12pt, a4paper]{article} % book, report, article, letter, slides
                                       % letterpaper/a4paper, 10pt/11pt/12pt, twocolumn/twoside/landscape/draft

%%%%%%%%%%%%%%%% PACKAGES %%%%%%%%%%%%%%%%%%%%%

\usepackage[utf8]{inputenc} % encoding

\usepackage[english]{babel} % use special characters and also translates some elements within the document.

\usepackage{amsmath}        % Math
\usepackage{amsthm}         % Math, \newtheorem, \proof, etc
\usepackage{amssymb}        % Math, extended collection
\usepackage{bm}             % $\bm{D + C}$
\newtheorem{theorem}{Theorem}[section]     % \begin{theorem}\label{t:label}  \end{theorem}<Paste>
\newtheorem{corollary}{Corollary}[theorem]
\newtheorem{lemma}[theorem]{Lemma}
\newenvironment{claim}[1]{\par\noindent\underline{Claim:}\space#1}{}
\newenvironment{claimproof}[1]{\par\noindent\underline{Proof:}\space#1}{\hfill $\blacksquare$}

\usepackage{hyperref}       % Hyperlinks \url{url} or \href{url}{name}

\usepackage{parskip}        % \par starts on left (not idented)

\usepackage{abstract}       % Abstract

\usepackage{graphicx}       % Images
\graphicspath{{./images/}}

\usepackage[vlined,ruled]{algorithm2e} % pseudo-code

% \usepackage[document]{ragged2e}  % Left-aligned (whole document)
% \begin{...} ... \end{...}   flushleft, flushright, center

%%%%%%%%%%%%%%%% CODE %%%%%%%%%%%%%%%%%%%%%

\usepackage{minted}         % Code listing
% \mint{html}|<h2>Something <b>here</b></h2>|
% \inputminted{octave}{BitXorMatrix.m}

%\begin{listing}[H]
  %\begin{minted}[xleftmargin=20pt,linenos,bgcolor=codegray]{haskell}
  %\end{minted}
  %\caption{Example of a listing.}
  %\label{lst:example} % You can reference it by \ref{lst:example}
%\end{listing}

\newcommand{\code}[1]{\texttt{#1}} % Define \code{foo.hs} environment

%%%%%%%%%%%%%%%% COLOURS %%%%%%%%%%%%%%%%%%%%%

\usepackage{xcolor}         % Colours \definecolor, \color{codegray}
\definecolor{codegray}{rgb}{0.9, 0.9, 0.9}
% \color{codegray} ... ...
% \textcolor{red}{easily}

%%%%%%%%%%%%%%%% CONFIG %%%%%%%%%%%%%%%%%%%%%

\renewcommand{\absnamepos}{flushleft}
\setlength{\absleftindent}{0pt}
\setlength{\absrightindent}{0pt}

%%%%%%%%%%%%%%%% GLOSSARIES %%%%%%%%%%%%%%%%%%%%%

%\usepackage{glossaries}

%\makeglossaries % before entries

%\newglossaryentry{latex}{
    %name=latex,
    %description={Is a mark up language specially suited
    %for scientific documents}
%}

% Referene to a glossary \gls{latex}
% Print glossaries \printglossaries

\usepackage[acronym]{glossaries} %

% \acrshort{name}
% \acrfull{name}
%\newacronym{kcol}{$k$-COL}{$k$-coloring problem}

\usepackage{enumitem}

%%%%%%%%%%%%%%%% HEADER %%%%%%%%%%%%%%%%%%%%%

\usepackage{fancyhdr}
\pagestyle{fancy}
\fancyhf{}
\rhead{Arnau Abella}
\lhead{Combinatorial Problem Solving}
\rfoot{Page \thepage}

%%%%%%%%%%%%%%%% TITLE %%%%%%%%%%%%%%%%%%%%%

\title{%
  Linear Programming: Box Wrappig
}
\author{%
  Arnau Abella \\
  \large{Universitat Polit\`ecnica de Catalunya}
}
\date{\today}

%%%%%%%%%%%%%%%% DOCUMENT %%%%%%%%%%%%%%%%%%%%%

\begin{document}

\maketitle

%%%%%%%%%%%%%%%%%%%%%%%

\section{Variables}\label{variables}

The following are the variables appearing in the linear equations of the model:

\begin{itemize}
  \setlength{\itemindent}{-.0in}
  \item[] $n$ = total number of boxes.
  \item[] $w$ = width of the wrapping.
  \item[] $l$ = maximum length of the wrapping.
  \item[] $length$ = current length of the wrapping (where $length \leq l$).
  \item[] $x^{tl}_i$ = x coordinate of the $i$-th box (${x^{tl} = \{0,\cdots, w-1\}}$).
  \item[] $y^{tl}_i$ = y coordinate of the $i$-th box (${y^{tl} = \{0,\cdots, l-1\}}$).
  \item[] $width_i$ = assigned width of the $i$-th box (${width = \{0,\cdots, w-1\}}$).
  \item[] $height_i$ = assigned height of the $i$-th box (${height = \{0,\cdots, l-1\}}$).
\end{itemize}

where $i = \{0, \cdots, n-1\}$.

%%%%%%%%%%%%%%%%%%%%%%%

\section{Constraints}\label{constraints}

(Bounds) Boxes must be placed inside the width of the wrapping:

\begin{equation}\label{eq:one}
  x^{tl}_i \leq w - witdth_i \quad \text{forall i}
\end{equation}

\newpage

(Rotation) The boxes must be placed in vertical or horizontal position:

\begin{equation}\label{eq:two}
  width_i + height_i = b\_width_i + b\_heigth_i
\end{equation}

where $b\_width_i$ and $b\_height_i$ are the given width and height of the $i$-th box.

(Overlapping) The boxes must not overlap:

\begin{equation}\label{eq:three}
  \begin{split}
    ((x^{tl}_i &\leq x^{tl}_j - width_i)\\
    + (x^{tl}_i &\geq x^{tl}_j + width_j)\\
    + (y^{tl}_i &\leq y^{tl}_j - height_i)\\
    + (y^{tl}_i &\geq y^{tl}_j + heigth_j)) \geq 1 \quad \text{for all} \ i, j, i \neq j
  \end{split}
\end{equation}

(Length) The current length of the wrapping must at least the position of the furthest box from the origin plus its height:

\begin{equation}\label{eq:four}
  length \geq y^{tl}_i + height_i \quad \text{forall i}
\end{equation}


%%%%%%%%%%%%%%%%%%%%%%%

\section{Objective function (to be minimized)}\label{variables}

\begin{equation}\label{eq:four}
  Cost = length
\end{equation}


%%%%%%%%%%%%%%%%%%%%%%%

\section{Optimizations}\label{optimizations}

\begin{itemize}
  \item The first box, by symmetry, can be placed on the left-side of the wrapping without changing the final result.
  \item Repeating branches of identical boxes is redundant, imposing an arbitrary order will prevent this.
  \item Square boxes can be placed indistinct vertically or horizontally.
\end{itemize}

%%%%%%%%%%%%%%%%%%%%

\end{document}
