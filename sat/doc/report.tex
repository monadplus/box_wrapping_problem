\documentclass[12pt, a4paper]{article} % book, report, article, letter, slides
                                       % letterpaper/a4paper, 10pt/11pt/12pt, twocolumn/twoside/landscape/draft

%%%%%%%%%%%%%%%% PACKAGES %%%%%%%%%%%%%%%%%%%%%

\usepackage[utf8]{inputenc} % encoding

\usepackage[english]{babel} % use special characters and also translates some elements within the document.

\usepackage{amsmath}        % Math
\usepackage{amsthm}         % Math, \newtheorem, \proof, etc
\usepackage{amssymb}        % Math, extended collection
\usepackage{bm}             % $\bm{D + C}$
\newtheorem{theorem}{Theorem}[section]     % \begin{theorem}\label{t:label}  \end{theorem}<Paste>
\newtheorem{corollary}{Corollary}[theorem]
\newtheorem{lemma}[theorem]{Lemma}
\newenvironment{claim}[1]{\par\noindent\underline{Claim:}\space#1}{}
\newenvironment{claimproof}[1]{\par\noindent\underline{Proof:}\space#1}{\hfill $\blacksquare$}

\usepackage{hyperref}       % Hyperlinks \url{url} or \href{url}{name}

\usepackage{parskip}        % \par starts on left (not idented)

\usepackage{abstract}       % Abstract

\usepackage{graphicx}       % Images
\graphicspath{{./images/}}

\usepackage[vlined,ruled]{algorithm2e} % pseudo-code

% \usepackage[document]{ragged2e}  % Left-aligned (whole document)
% \begin{...} ... \end{...}   flushleft, flushright, center

%%%%%%%%%%%%%%%% CODE %%%%%%%%%%%%%%%%%%%%%

\usepackage{minted}         % Code listing
% \mint{html}|<h2>Something <b>here</b></h2>|
% \inputminted{octave}{BitXorMatrix.m}

%\begin{listing}[H]
  %\begin{minted}[xleftmargin=20pt,linenos,bgcolor=codegray]{haskell}
  %\end{minted}
  %\caption{Example of a listing.}
  %\label{lst:example} % You can reference it by \ref{lst:example}
%\end{listing}

\newcommand{\code}[1]{\texttt{#1}} % Define \code{foo.hs} environment

%%%%%%%%%%%%%%%% COLOURS %%%%%%%%%%%%%%%%%%%%%

\usepackage{xcolor}         % Colours \definecolor, \color{codegray}
\definecolor{codegray}{rgb}{0.9, 0.9, 0.9}
% \color{codegray} ... ...
% \textcolor{red}{easily}

%%%%%%%%%%%%%%%% CONFIG %%%%%%%%%%%%%%%%%%%%%

\renewcommand{\absnamepos}{flushleft}
\setlength{\absleftindent}{0pt}
\setlength{\absrightindent}{0pt}

%%%%%%%%%%%%%%%% GLOSSARIES %%%%%%%%%%%%%%%%%%%%%

%\usepackage{glossaries}

%\makeglossaries % before entries

%\newglossaryentry{latex}{
    %name=latex,
    %description={Is a mark up language specially suited
    %for scientific documents}
%}

% Referene to a glossary \gls{latex}
% Print glossaries \printglossaries

\usepackage[acronym]{glossaries} %

% \acrshort{name}
% \acrfull{name}
%\newacronym{kcol}{$k$-COL}{$k$-coloring problem}

\usepackage{enumitem}

%%%%%%%%%%%%%%%% HEADER %%%%%%%%%%%%%%%%%%%%%

\usepackage{fancyhdr}
\pagestyle{fancy}
\fancyhf{}
\rhead{Arnau Abella}
\lhead{Combinatorial Problem Solving}
\rfoot{Page \thepage}

%%%%%%%%%%%%%%%% TITLE %%%%%%%%%%%%%%%%%%%%%

\title{%
  SAT: Box Wrappig
}
\author{%
  Arnau Abella \\
  \large{Universitat Polit\`ecnica de Catalunya}
}
\date{\today}

%%%%%%%%%%%%%%%% DOCUMENT %%%%%%%%%%%%%%%%%%%%%

\begin{document}

\maketitle

%%%%%%%%%%%%%%%%%%%%%%%

\section{Variables and Constraints}%
\label{sec:Variables and Clauses}

The following constants are used in the box wrapping problem:

\begin{itemize}
  \item $1 \leq W \leq 11$: width of the roll.
  \item $1 \leq L \leq (\sum_{b \in B} h_b)$: maximum length of the roll.
  \item $1 \leq B \leq 13$: total number of boxes.
  \item $w_{b \in B}$: width of the $b$-th box.
  \item $h_{b \in B}$: height of the $b$-th box.
\end{itemize}

The following variables are used in the problem:

\begin{itemize}
  \item $tl_{bij}$ where $i \in W, j \in L, b \in B$: box represented by its top-left coordinate.
  \item $a_{bij}$ where $i \in W, j \in L, b \in B$: area of each box. Overlapping of areas is not allowed.
  \item $r_b$ where $b \in B$: rotation of each box. \textit{True} means the box is rotated.
\end{itemize}

The following constraints are used in the problem:

\begin{itemize}
  \item \textbf{Exactly One}. Each box must appear once and only once in the paper roll.

  \begin{equation}\label{eq:1}
    \bigwedge\limits_{b=1}^{B} ((\sum_{i=1}^{W} \sum_{j=1}^{L} tl_{bij}) = 1)
  \end{equation}

  \item \textbf{Bounds}. Each box must be inside the bounds of the roll.

  \begin{equation}\label{eq:2}
    \begin{aligned}
      &\forall i_{\in W} \forall j_{\in L} \bigwedge\limits_{b=1}^{B} \neg tl_{bij} & \quad w_b = h_b, i+w_b > W, j + h_b > L\\
      &\forall i_{\in W} \forall j_{\in L} \bigwedge\limits_{b=1}^{B} (\neg tl_{bij} \land r_b) \land (\neg tl_{bij} \lor \neg r_b) & \quad w_b \neq h_b, i+w_b > W, j + h_b > L
    \end{aligned}
  \end{equation}

  \item \textbf{Overlapping}. This constrant is divided in two constraints:

    \begin{itemize}
      \item Clauses to represent the \textbf{area} of each box.

        \begin{equation}\label{eq:3}
          \begin{aligned}
            &\forall i_{\in W} \forall j_{\in L} \bigwedge\limits_{b=1}^{B} (\neg tl_{bij} \lor a_{bij}) \land \neg r_b & \quad w_b = h_b \\
            &\forall i_{\in W} \forall j_{\in L} \bigwedge\limits_{b=1}^{B} (\neg tl_{bij} \lor a_{bij} \lor r_b) \land (\neg tl_{bij} \lor a_{bij} \lor \neg r_b) & \quad w_b \neq h_b
          \end{aligned}
        \end{equation}

      \item \textbf{At most one constraint} of these areas to prevent the \mbox{overlapping}.

        \begin{equation}\label{eq:4}
          \forall b_{\in B} \sum_{i=1}^{W} \sum_{j=1}^{L} a_{bij} \leq 1
        \end{equation}

        Notice, this constraint is encoded using the \mbox{\textit{at most one logarithmic encoding}}.

    \end{itemize}

  \item (Optimization) By symmetry, placing the box on the left-half side is the same as placing the box in the right-half side. This constraint forces the biggest box to be on the left-half side, in the $(0,0)$ coordinate.

    \begin{equation}\label{eq:5}
      \begin{aligned}
        &tl_{b00} \land (\bigwedge\limits_{i=1,j=1}^{W,L} \neg tl_{bij}) \quad b \in B \\
      \end{aligned}
    \end{equation}
\end{itemize}

\newpage

\section{Implementation}%
\label{sec:Implementation}

This implementation is based on mios, a minisat-based CDCL SAT solver written in purely Haskell \cite{mios}. It is one of the few, open-source, SAT solvers written in Haskell with high performance.

The implementation is split in two files:

\begin{itemize}
  \item \code{app/Main.hs}: main loop, at each iteration the objective function is minimized.
  \item \code{src/SAT.hs}: implementation of the constraints, clauses and custom encoding.
\end{itemize}

Let's have a look at the encoding of the constraints:

\begin{itemize}

  \item Constraint \ref{eq:1}
    \begin{listing}[H]
      \inputminted[firstline=275, lastline=279, breaklines]{haskell}{../src/SAT.hs}
      \inputminted[firstline=218, lastline=222, breaklines]{haskell}{../src/SAT.hs}
      \inputminted[firstline=223, lastline=226, breaklines]{haskell}{../src/SAT.hs}
    \end{listing}

    \begin{listing}[H]
      \inputminted[firstline=239, lastline=253, breaklines]{haskell}{../src/SAT.hs}
    \end{listing}

  \item Constraint \ref{eq:2}
    \begin{listing}[H]
      \inputminted[firstline=281, lastline=295, breaklines]{haskell}{../src/SAT.hs}
    \end{listing}

  \newpage

  \item Constraint \ref{eq:3}
    \begin{listing}[H]
      \inputminted[firstline=297, lastline=315, breaklines]{haskell}{../src/SAT.hs}
    \end{listing}

  \item Constraint \ref{eq:4}
    \begin{listing}[H]
      \inputminted[firstline=346, lastline=352, breaklines]{haskell}{../src/SAT.hs}
    \end{listing}

  \newpage

  \item Constraint \ref{eq:5}
    \begin{listing}[H]
      \inputminted[firstline=265, lastline=272, breaklines]{haskell}{../src/SAT.hs}
    \end{listing}

\end{itemize}


%%%%%%%%%%%%%%%%%%%%%%%


\begin{thebibliography}{1}
  \bibitem{mios} Narazaki Shuji. A Minisat-based CDCL SAT solver in Haskell. url{https://hackage.haskell.org/package/mios-1.6.2}
\end{thebibliography}


%%%%%%%%%%%%%%%%%%%%%%%

\end{document}
